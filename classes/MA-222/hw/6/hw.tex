\documentclass[12pt, letterpaper]{article}
\usepackage{comment}
\usepackage[utf8]{inputenc}
\usepackage[a4paper, total={7in, 10in}]{geometry}
\usepackage{amsmath}
\usepackage{tikz}
\usepackage{pgfplots}
\usepackage{setspace}

\title{HW 6}
\author{Ben Lirio}
\date{April 2}
\renewcommand{\baselinestretch}{1.5}

\begin{document}
\maketitle
\textit{I pledge my honor that I have abided by the Stevens Honor System}
\newpage
%Problem 1

\section{}
\subsection{}
$P(X \ge 4000) = 1 - F(4000) = \textbf{0.11931}$
\newline
$P(3000 \le X \le 4000) = F(4000) - F(3000) = \textbf{0.69563}$
\newline
I Used Julia(julialang.org) for all normal distribution cdf calculations:
\begin{verbatim}
mean = 3432
std = 482
d = Normal(mean, std)
cdf.(d, 4000)
> 0.11931
\end{verbatim}
\subsection{}
$P(X \le 2000 \cap X \ge 5000) = F(2000) + (1 - F(5000)) = \textbf{0.002}$
\subsection{}
\(
7\ lb = 7\ lb\cdot 453.582\ lb/g = 3175.144\ g \\
P(X \ge 3175.144) = 1 - F(3175.144) = \textbf{0.7029} \\
\)
\subsection{}
Any weight below $z = 0.0005$ or above $z = 0.9995$ is considered extreme.
\newline
\(
c = \textrm{0.05th percentile} = 1845.96611 \\
c = \textrm{99.95th percentile} = 5018.03 \\
\textrm{So anything} \le \textbf{1845.966} \ \textrm{or} \ge \textbf{5018.03} \\
\)
I Used Julia(julialang.org) for all normal distribution quantile calculations:
\begin{verbatim}
mean = 3432
std = 482
d = Normal(mean, std)
quantile.(d, 0.0005)
> 1845.96611
\end{verbatim}
\subsection{}
\(
\mu = 3432 \cdot 0.00220462 = 7.56625584 \\
\sigma = 482 \cdot 0.00220462 = 1.06262684 \\
P(X \ge 7) = 1 - F(7) = \textbf{0.7029} \\
\)
I am getting the same answer as 1.3.

% Problem 2
\newpage
\section{}
Given data distrubuted binomial with $n = 200$ and $p = 0.1$ I clearly see $0.1 \cdot 200 = 20 \ge 10$ and $0.9 \cdot 200 = 180 \ge 10$,
therefore I will approximate using a Gaussian distribution with
\newline
\(
\mu = 200 \cdot 0.1 = 20 \\
\sigma = \sqrt{200\cdot 0.1 \cdot 0.9} = 4.2426 \\
\)
\subsection{}
I am including 30 \\
$P(X \le 30.5) = F(30.5) = \textbf{0.99333}$
\subsection{}
I am not including 30 \\
$P(X \le 29.5) = F(29.5) = \textbf{.987427}$
\subsection{}
Add $0.5$ to the larger value and subtract $0.5$ from the smaller \\
$P(14.5 \le X \le 25.5) = F(25.5) - F(14.5) = \textbf{.80514}$

% Problem 3
\newpage
\section{}
\subsection{}
The Event $E = {X \ge t}$ can be viewed in terms of $A_i$ as the event
\newline
\(
A_1 \cap A_2 \cap A_3 \cap A_4 \cap A_5 \\
\)
\subsection{}

\(
P(X \ge t) = P(A_2 \ge t) \cdot P(A_2 \ge t) \cdot P(A_3 \ge t) \cdot P(A_4 \ge t) \cdot P(A_5 \ge t) \\
\)
Each $P(A_i) = e^{-0.01t}$ so the numerical representation of $P(X \ge t)$ is
\newline
\(
(e^{-0.01t})^5 = e^{-0.05t} \\
\)
Now it is trivial to see the cdf at $t$ is the complement of $P(X \ge t)$ so
\newline
\(
F(t) = 1 - P(X \ge t) = 1 - e^{-0.05t}\\
\)
\[
F(t) = 
\begin{cases}
1 - e^{-0.05t} & x \ge 0 \\
0 & \textrm{otherwise} \\
\end{cases}
\]
Now to get the pdf all I need to do is take the derivative of the cdf \\
\(
f(t) = \frac{d}{dt}F(t) \\
= 0.05e^{-0.05t} \\
\)
\[
f(t) =
\begin{cases}
0.05e^{-0.05t} & x \ge 0 \\
0 & \textrm{otherwise} \\
\end{cases}
\]
\newline
Consequently $X$ has an exponential distribution
\subsection{}
Due to the nature of exponents, namly $e^{x} \cdot e^{y} = e^{x+y}$, combining $n$
exponential r.v.'s $X$ will also be exponential.

% Problem 4
\newpage
\section{}
Given data distrubuted binomial with $n = 250$ and $p = 0.05$ I clearly see $0.05 \cdot 250 = 12.5 \ge 10$ and $0.95 \cdot 250 = 237.5 \ge 10$,
therefore I will approximate using a Gaussian distribution with \\
\(
\mu = 250 \cdot 0.05 = 12.5 \\
\sigma = sqrt{250 \cdot 0.05 \cdot 0.95} = 3.44601 \\
\)
\subsection{}
$10\%$ of the boards is equivalent to 25, thus \\
\(
P(X \ge 24.5) = 1 - F(24.5) = \textbf{0.000248} \\
\)
Note: I am using $24.5$ as opposed to $25$ because $25$ is a whole number,
would this still hold for $11\%$ making $27.5$ become $27$? Or do I round down
to $27$ first then subtract the $0.5$? 
\subsection{}
When approximating binomials with the Gausian distribution, I take 0.5 on either side of the discrte value, so \\
\(
P(9.5 \le X \le 10.5) = F(10.5) - F(9.5) = \textbf{0.08883} \\
\)

\newpage
\section{}
\subsection{}
\begin{tikzpicture}
\begin{axis}[
    axis lines = left,
    xlabel = $x$,
    ylabel = {$f(x)$},
]
%Below the red parabola is defined
\addplot [
    domain=0:25,
    samples=100,
    color=red,
]
{0.1*exp(-.2*x)};
%Here the blue parabloa is defined
\addplot [
    domain=-25:0,
    samples=100,
    color=red,
    ]
    {0.1*exp(.2*x)};
\addlegendentry{$0.1e^{0.2|x|}$}
\end{axis}
\end{tikzpicture}
\newline
Integrating, I get
\(
\int_{-\infty}^{0} 0.1e^{0.2\cdot x}\ dx + \int_{0}^{\infty} 0.1e^{-0.2 \cdot x}\ dx \\
= .1/.2 + .1/.2 \\
= .5 + .5 \\
= 1 \\
\)

\subsection{}
\(
\int_{-\infty}^{\infty} f(x) dx \\
F(x) = 
\begin{cases}
0.5e^{0.2 \cdot x} & x \le 0 \\
1 + -0.5e^{-0.2 \cdot x} & x > 0 \\
\end{cases}
\)
\newline
\begin{tikzpicture}
\begin{axis}[
    axis lines = left,
    xlabel = $x$,
    ylabel = {$F(x)$},
]
%Below the red parabola is defined
\addplot [
    domain=0:25,
    samples=100,
    color=red,
]
{-0.5*exp(-0.2*x) + 1};
%Here the blue parabloa is defined
\addplot [
    domain=-25:0,
    samples=100,
    color=red,
    ]
    {0.5*exp(.2*x)};
\end{axis}
\end{tikzpicture}
\subsection{}
\(
P(X \le 0) = F(0) = 0.5e^{0.5 \cdot 0} \\
= \textbf{0.5} \\
P(X \le 2) = F(2) = 1 + -0.5^{-0.2 \cdot 2} \\
= \textbf{0.66484} \\
P(-1 \le X \le 2) = F(2) - F(-1) = 0.66484 - 0.5e^{0.2 \cdot -1} \\
= 0.66484 - 0.409365 \\
= \textbf{0.255475} \\
\)
The probability that an error of more than 2 miles is made can be interpreted as $P(X \ge 2)$.
Thus, using the following calculations, $1 - F(2) = \textbf{0.33516}$ is the probability.
\end{document}
