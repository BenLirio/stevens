\documentclass[12pt, letterpaper]{article}
\usepackage{comment}
\usepackage[utf8]{inputenc}
\usepackage[a4paper, total={7in, 10in}]{geometry}
\usepackage{amsmath}
\usepackage{tikz}
\usepackage{pgfplots}
\usepackage{setspace}

\title{HW 5}
\author{Ben Lirio}
\date{March 22}

\renewcommand{\baselinestretch}{1.5}

\begin{document}
\maketitle
\textit{I pledge my honor that I have abided by the Stevens Honor System}
\newline
*I used a calculator for any integrals or derivites that required more than the power rule.
\section{}
\subsection{}
Plot of $p(y)$
\begin{center}
\begin{tikzpicture}
\begin{axis}[
	xlabel = $y$,
	ylabel = {pd},
]
\addplot[
color=red,
domain=0:5
]{x/25};
\addplot[
color=red,
domain=5:10
]{2/5 - x/25};
\addplot[
color=red,
domain=-2:0
]{0};
\addplot[
color=red,
domain=10:12
]{0};
\end{axis}
\end{tikzpicture}
\end{center}
\subsection{}
Integral $\int_{-\infty}^{\infty} f(y) \,dy$
\newline
\newline
$ = \int_{-\infty}^{0} 0 \, dy +$ 
$ \int_{0}^{5} \frac{y}{25} \, dy +$
$ \int_{5}^{10} \frac{2}{5} - \frac{y}{25} \, dy +$
$ \int_{10}^{\infty} 0 \, dy$ 
\newline
\newline
$ = 0 + \frac{5^{2}}{50}$
$ +(\frac{2\cdot 10}{5}-\frac{10^{2}}{50})$
$ - (\frac{2\cdot 5}{5}-\frac{5^{2}}{50})$
\newline
\newline
$ = \frac{25}{50}$
$ + (\frac{20}{5} - \frac{100}{50})$
$ - (\frac{10}{5} - \frac{25}{50})$
\newline
\newline
$ = 0.5 + (4-2) - (2-0.5)$
$ = 0.5 + 2 - 1.5 $
$ = $ \textbf{1}
\subsection{}
The probabity of waiting at most 3 minutes can be calculated using the cumulative distrabution function.

$ \int_{-\infty}^{3} f(y) \,dy$
$ = \int_{0}^{3} \frac{y}{25} \,dy$
$ = \frac{9}{50} $
$ = $ \textbf{18\%}

\subsection{}
The probability of waiting at most 8 minutes can be calculated using the cumulative distrabution fuction with $Y = 8$.

$ \int_{-\infty}^{8} f(y) \,dy$
$ = \int_{0}^{5} \frac{y}{25} \,dy + \int_{5}^{8} \frac{2}{5} - \frac{y}{25} \,dy$
\newline
\newline
$ = \frac{25}{50} + \frac{21}{50} = \frac{46}{50}$
$ = $ \textbf{92\%}

\subsection{}

The probability $3 \le Y \ge 8$ can simply be calculated by subtracting $P(Y \le 4)$ from $P(Y \le 8)$
Using the calcuations from answers 1.3 and 1.4, I get $.92 - .18 = .78 = $ \textbf{74\%}
\subsection{}
The probability $Y \le 2$ or $Y \ge 6$ can be calcuated as follows:
\[
	\int_{0}^{2} \frac{y}{25}\,dy + \int_{6}^{10} \frac{2}{5} - \frac{y}{25}\,dy
\]
\(
\dfrac{2^{2}}{50} + (\dfrac{20}{5} - \dfrac{10^{2}}{50}) - (\dfrac{12}{5} - \dfrac{6^{2}}{50}) \\
\\
\dfrac{2}{25} + \dfrac{8}{5} - \dfrac{64}{50} \\
\\
\dfrac{4+80-64}{50} \\
\\
\dfrac{20}{50} \\
\\
\textbf{40\%}
\)
\section{}
\subsection{}

\begin{center}
\begin{tikzpicture}
\begin{axis}[
	xlabel = $y$,
	ylabel = {pd},
]
\addplot[
color=red,
domain=0:1
]{90*x*x*x*x*x*x*x*x*(1 - x)};
\end{axis}
\end{tikzpicture}
\end{center}

\begin{center}
\begin{tikzpicture}
\begin{axis}[
	xlabel = $y$,
	ylabel = {cd},
]
\addplot[
color=red,
domain=0:1
]{10*x*x*x*x*x*x*x*x*x - 9*x*x*x*x*x*x*x*x*x*x};
\end{axis}
\end{tikzpicture}
\end{center}
\subsection{}
\(
\int_{-\infty}^{.5} f(y)\, dy \\
= \int_{0}^{.5} 90x^{8} (1 - x)\, dy \\
= \frac{90\cdot .5^{9}}{9} - \frac{90\cdot .5^{10}}{10} \\
= 10\cdot .5^{9} - 9\cdot .5^{10} \\
= .019531 - .00879 \\
= .010742 \\
\textbf{1.07\%}
\)
\subsection{}
First, $P(.25 < X \le .5) \equiv P(.25 \le X \le .5)$ so I will only solve the second one.
\newline
CDF = $10x^{9} - 9x^{10}$
\newline
\(
(10(.5)^{9} - 9(.5)^{10}) - (10(.25)^{9} - 9(.25)^{10}) \\
\textbf{1.07\%} \\
\)
\subsection{}
\(
.75 = 10x^{9} - 9x^{10} \\
0.903596 \\
\textbf{90.36\%}
\)
\subsection{}
\(
E(X) = \int_{-\infty}^{\infty} xf(x)\,dx \\
= \int_{0}^{1} 90x^{9}(1 - x)\, dx \\ 
= 9(1)^{10} - \frac{90(1)^{11}}{11} \\
= 9 - \frac{90}{11} \\
= \frac{9}{11} \\
E(X) = \textbf{.8182} \\
\)
\newline
\(
\sigma_{X}^{2} = V(X) = \int_{-\infty}^{\infty} x^{2}f(x)\,dx - E(X)^{2} \\
= \int_{0}^{1} 90x^{10}(1 - x)\, dx [E(X)]^{2}\\
= \frac{90x^{11}}{11} - \frac{90x^{12}}{12} [E(X)]^{2} \\
= \frac{90}{11} - \frac{90}{12} - [E(X)]^{2} \\
= \frac{15}{12} - [E(X)]^{2} \\
= .6818 - .8182^{2}\\
V(X) = 0.012366 \\
\sigma_{X} = \textbf{.111207} \\
\)
\subsection{}
\(
1 - P(.707 \le X \le .929) = 1 - \int_{.707}^{.929} 90x^{8}(1 - x)\, dx \\
= 1 - ((10(.929)^{9} - 9(.929)^{10}) - (10(.707)^{9} - 9(.707)^{10})) \\
= 1 - .684219 \\
= \textbf{.31} \\
\)
\section{}
\subsection{}
\(
P(X \le 1) = F(1) \\
= \frac{1}{4}\cdot \Big[ 1 + ln\big(\frac{4}{1}\big) \Big] \\
= \textbf{.596574} \\
\)
\subsection{}
\(
P(1 \le X \le 3) = F(3) - F(1) \\
= \frac{3}{4}\cdot \Big[1+ln\big(\frac{4}{3}\big) \Big] - F(1) \\
= 0.9657 - .596574 \\
= \textbf{.369126} \\
\)
\subsection{}
\(
f(x) = \frac{d}{dx}[F(x)] \\
= \frac{ln(4/x)}{4} \\
\)
But the function is only defined for $0 < x \le 4$
\newline
\(
p(x) = 
\begin{cases}
	\frac{ln(4/x)}{4} & 0 < x \le 4 \\
	0 & \textrm{otherwise} \\
\end{cases}
\)
\section{}
\subsection{}
\(
k = 1/\big(\int_{0}^{\infty} (1 + x/2.5)^{-7}\big) \\
= \frac{1}{0.416666} \\
k = \textbf{2.4} \\
\)
\subsection{}
\begin{center}
\begin{tikzpicture}
\begin{axis}[
	xlabel = $y$,
	ylabel = {pd},
]
\addplot[
color=red,
domain=0:3
]{2.4*1/((1+x/2.5)*(1+x/2.5)*(1+x/2.5)*(1+x/2.5)*(1+x/2.5)*(1+x/2.5)*(1+x/2.5))};
\end{axis}
\end{tikzpicture}
\end{center}
\subsection{}
\(
E(X) = \int_{-\infty}^{\infty} x2.4(1 + \frac{x}{2.5})^{-7} \\
= \int_{0}^{\infty} x2.4(1 + \frac{x}{2.5})^{-7} \\
E(X) = \textbf{0.5} \\
\)
\(
V(X) = E(X^{2}) - [E(X)]^{2} \\
= \int_{-\infty}^{\infty} x^{2}2.4(1 + \frac{x}{2.5})^{-7} - [E(X)]^{2} \\
= 0.625 - .5^{2} \\
V(X) = 0.375 \\
\sigma_{X} = \sqrt{V(X)} \\
\sigma_{X} = \sqrt{0.375} = \textbf{0.612372} \\
\)
\subsection{}
Using the law of the unconcious statistition, I can use $f(x)$ in my computation of $E(Y)$ by instead integrating over $f(y)$ and adding the correct coeficient.
\newline
\(
E(Y) = \int_{.5}^{10.5} .8yf(y)\, dy + \int_{10.5}^{\infty} yf(y)\, dy \\
= \int_{.5}^{10.5} .8(y-.5)(2.4(1 + y/2.5)^{-7})\, dy +
\int_{10.5}^{\infty} (y-10.5)(2.4(1 + y/2.5)^{-7})\, dy \\
= 0.160241 + 0.000132 \\
E(Y) = \textbf{0.160373} \\
\)
Insurance companies should expect to pay about $160\$$ per client.
\newline
\end{document}
