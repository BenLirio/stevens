\documentclass{article}
\usepackage{blindtext}
\usepackage{amsmath}
\title{HW 6}
\author{Ben Lirio \thanks{I pledge my honor that I have abided by the Stevens Honor System}}
\begin{document}
\begin{titlepage}
\maketitle
\end{titlepage}
\section{Exercise}
$x^{2} + 7x + 1 \equiv_{11} 0$
\newline
$ = x^{2} + 7x + 12 \equiv_{11} 0$
\newline
$ = (x+4)(x+3) \equiv_{11} 0$
\newline
Because $11$ is prime, the congruence is valid for any $x$ such that $11|(x+4)$ or $11|(x+3)$.
\newline
$x + 4 \equiv_{11} 0$
\newline
$x + 3 \equiv_{11} 0$
\newline
which can also be written as
\newline
$x \equiv_{11} 7$
\newline
$x \equiv_{11} 8$
\newline
So it consists of values $x = 11y_{1} + 7$ and $x = 11y_{2} + 8$ for any $y1, y2 \in Z$

\section{}
I notice that $x^{2} \equiv_{35} 11$ is of the form. $x^{2} \equiv_{pq} 11$ where $p = 5$ and $q = 7$.
\newline
$x^{2} \equiv_{5} 1$
\newline
$x^{2} \equiv_{7} 4$
\newline
Can then be written as
\newline
\[
	\begin{cases}
	x \equiv_{5} \pm 1 \\
	x \equiv_{7} \pm 2 \\
	\end{cases}
\]
Using the Chinese remainder theorem:
\newline
\(
x = 1 + 5y \\
1 + 5y \equiv_{7} 2 \\
5y \equiv_{7} 1 \\
y \equiv_{7} 3 \\
x = 1 + 5\cdot 3 \\
x = 16
\)
I will set $\alpha_{1} \equiv_{35} \pm 16$
\newline
\(
x = 4 + 5y \\
4 + 5y \equiv_{7} 2 \\
5y \equiv_{7} 5 \\
y \equiv_{7} 1 \\
x = 4 + 5\cdot 1 \\
x = 9
\)
I will set $\alpha_{2} \equiv_{35} \pm 9$
\newline
$\{\alpha_{1} = 16, -\alpha_{1} = 19, \alpha_{2} = 9, -\alpha_{2} = 26\}$
\section{Exercise}
$(19/23) = -(23/19) = -(4/19) = -1$
\newline
$(18/43) = (9/43)(2/43) = 1\cdot -1 = -1$
\section{Exercise}
Rewriting the following as $(r/p) = 1$ and $(ab/p) = 1$. I will use the easy properties of Legrandre symbols.

$(ab/p) = (a/p)(b/p) = 1$ Legrange symbols range the values $\{0,1,-1\}$, and there are only
two combinations $1 \cdot 1 = 1$ and $-1 \cdot -1 = 1$ that satisfy the equation. So
$a$ and $b$ must either both be residues or neither can be residues.

\section{Exercise}
\textbf{Remote Coin Toss}

\textbf{Alice}

randomly selects 2 primes $p = 47, q = 79$. Then computes $n = 47\cdot 79 = 3713$

Then sends $n$ to Bob

\textbf{Bob}

recieves $n = 3713$ and chooses a random value $x = 123$, then calculates
$123^{2} \equiv_{3713} 277$

Bob then sends $x^{2}$ to Alice

\textbf{Alice}

Upon recieving $277$ from Bob, Alice find all four roots of $277$ mod 3713.

$x^{2} \equiv_{3713} 277$ Because Alice has the roots of 3713 she can rewrite the equation as

\[ \begin{cases}
	x^{2} \equiv_{47} 42 \\
	x^{2} \equiv_{79} 40 \\
	\end{cases}
\]

Using $x = a^{\frac{p+1}{4}}$ I calculate

$42^{\frac{47+1}{4}} \% 47 = 18$
$47^{\frac{79+1}{4}} \% 79 = 44$

Then use the Chinese remiander theorem to solve

\[ \begin{cases}
	x \equiv_{47} \pm 18 \\
	x \equiv_{79} \pm 44 \\
	\end{cases}
\]

$\alpha_{1} = \pm 676$

$\alpha_{2} = \pm 123$

Now, here is where the 'flip' happens,
Alice has the choice to send $\pm 676$ or $\pm 123$ to Bob.

If Bob is able to factor $n$ he wins. So if Alice chooses to send one of $\pm 676$ to Bob, he will recieve enough information to factor $n$.
On the other hand, if Alice sends one of $\pm 123$ to Bob, he will recieve redundent information and
be unable to factor $n$.
\newline
So $3037$ and $676$ represent loosing calls for Alice.
And, $123$ and $3590$ represent winning calls for Alice.

\end{document}
