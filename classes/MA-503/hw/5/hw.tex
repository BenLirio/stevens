\documentclass{article}
\usepackage[table]{xcolor}
\usepackage{amssymb}

\definecolor{myGray}{rgb}{.9,.9,.9}
\begin{document}
Ben Lirio

I plege my honor that I have abided by the Stevens Honor System.
\setcounter{section}{4}
\section{Homework}
\subsection{Excercise}
\begin{center}
\rowcolors{1}{white}{myGray}
\begin{tabular}{|ll|}
\hline
	$x$ & $(2^{x})\%29$  \\
\hline
	0 & 1 \\
	1 & 2 \\
	2 & 4 \\
	3 & 8 \\
	4 & 16 \\
	5 & 3 \\
	6 & 6 \\
	7 & 12 \\
	8 & 24 \\
	9 & 19 \\
	10 & 9 \\
	11 & 18 \\
	12 & 7 \\
	13 & 14 \\
	14 & 28 \\
	15 & 27 \\
	16 & 25 \\
	17 & 21 \\
	18 & 13 \\
	19 & 26 \\
	20 & 23 \\
	21 & 17 \\
	22 & 5 \\
	23 & 10 \\
	24 & 20 \\
	25 & 11 \\
	26 & 22 \\
	27 & 15 \\
\hline
\end{tabular}
\end{center}
$log_{2}(21) = 17$
\subsection{Excercise}
In CDH the following equations are true
\newline
$g^a \equiv A\ (\textrm{mod}\ n)$
\newline
$g^b \equiv B\ (\textrm{mod}\ n)$
\newline
$A^b \equiv B^a \equiv g^{ab}\ (\textrm{mod}\ n)$
\newline
Therefore when $n = 29$, $A = 18$, $B = 14$, the following must be true. $K \equiv 18^b \equiv 14^a \equiv 2^{log_2(18)log_2(14)}\ (\textrm{mod}\ n)$. According to the previous table $log_2(18) = 11$ and $log_2(14) = 13\ \textrm{mod}\ 29$. Consequently $2^{(11)(13)} \equiv 2^{143}\ \textrm{mod}\ 29$. Using Fermat's little theorem, I notice $2^{28} \equiv 1\ \textrm{mod}\ 29$. Which simplifies the equation to $2^{3} \equiv 8\ \textrm{mod}\ 29$. Leading to $K =$ \textbf{8}.
\subsection{Excercise}
Given $n = 29$, $g = 2$, $A = 17$, $c_1 = 6$, $c_2 = 10$, one way for Eve to compute the message is for her find the ephemeral key, I'll call $k$. This key was used to generate both $c_1$, and $c_2$ as follows, $c_1 \equiv g^{k}\ \textrm{mod}\ 29$ and $c_2 \equiv mA^{k}\ \textrm{mod}\ 29$. Therefore $k = log_2(c_1) \equiv log_2(6)\equiv 6\ \textrm{mod}\ 29$. Using $k = 6$. Next I would like to find $(A^{k})^{-1}$ in order to calculate $(A^{k})^{-1}c_2 \equiv mA^k(A^{k})^{-1} \equiv m\ \textrm{mod}\ 29$
\subsection{Excercise}
I will first create two generating functions using the given values, one I will call babystep, and the other I will call giant step. Babystep is $(e)g^{i}$ where $e$ is the multiplicative identity, $g$ is the generator and $i$ is the current generation step. The giant step function will be $h^{-in}$ where $h$ is the value I am taking the log of and $n = floor(\sqrt{N}) + 1$ where $N$ is the order.

$N = 36$

$g = 2$

$h = 3$

$e = 1$

$n = 7$

Note: one helpful precomputation is $u = g^{-n} = 24$

\begin{center}
\rowcolors{1}{white}{myGray}
\begin{tabular}{|lll|}
\hline
	i,j & baby-step $(e)g^{i}$ & giant-step $(h)g^{-jn}$ \\
\hline
0 & 1 & 3 \\
1 & 2 & 35 \\
2 & 4 & 26 \\
3 & 8 & 32 \\
4 & 16 & 28 \\
5 & 32 & 6 \\
\hline
\end{tabular}
\end{center}
Match found on $i = 5$ and $j = 3$. $(1)2^{5} \equiv (3)2^{(-3)(7)} \equiv\ \textrm{mod}\ 37$

Now if I multiply both sides by the $g^{-jn}$ the right side will simplify to just $h = 3$ and the left side will be a power of $g$. $2^{26} \equiv 3\ (\textrm{mod}\ 37)$. So $m =$ \textbf{3}
\subsection{Excercise}
First I notice that $\phi(37) = 36 = 2^{2} 3^{2}$. So I will solve two concruencies, one $\textrm{mod}\ 2^{2}$ and the other $\textrm{mod}\ 3^{2}$. Using the the formula. For simplicity, let $c_i = p_i^{e_i}$ where $p_i$ is a prime factor and $e_i$ is the exponent of that factor.

$19^{\frac{\phi(37)}{c_i}} \equiv 2^{\frac{\phi(37)r}{c_i}}\ (\textrm{mod}\ 37)$

Where $r$ is bound by $0 \le r < c_i$

$19^{9} \equiv 2^{9(r)}\ (\textrm{mod}\ 37)$ is valid for $r = 3$ therefore $x \equiv 3\ (\textrm{mod}\ 4)$

$19^{4} \equiv 2^{4(r)}\ (\textrm{mod}\ 37)$ is valid for $r = 8$ therefore $x \equiv 8\ (\textrm{mod}\ 9)$

Using the Chinese Remainder Theorem I can calculate $x$ as follows: Using the first congrunce I get $x = 3 + 4y$. Now I can plug this into the second formula to obtain $3 + 4y \equiv 8 (\textrm{mod}\ 9)$. Which can be simplified down to, $4y \equiv 5 (\textrm{mod}\ 9)$. Using Extended Euclidean algorithm, I find the multiplicative inverse of $4 (\textrm{mod}\ 9)$ is $7$. So $y \equiv 35 \equiv 8 (\textrm{mod}\ 9)$. Now using the equivalance $y = 8$, I can plug it back into the relation and get $x = 3 + 4(8) = 35$. Therefore $log_2{19} = 35\ (\textrm{mod}\ 37)$
\subsection{Excercise} I wasted too much time making tables in LaTeX and ran out of time.
\subsection{Excercise}
\begin{itemize}
	\item $(\mathbb{Z}, +, \cdot)$
	\begin{itemize}
		\item (R1) Associativity $(a + b) + c = a + (b + c)$, Identity $a + 0 = a$, Invertability $a - a = 0$, Commutativity, $a + b = b + a$. Since all of these are valid for $(\mathbb{Z}, +)$, (R1) passes.
		\item (R2) Multiplication is associative, meaning $(a * b) * c = a * (b * c)$ which is trivialy true, and the identify $1 \in \mathbb{Z}$ is also true. (R2) passes.
		\item (R3) Although not a complete proof, I will give an example of $(a + b)c = ac + bc$ given $a, b, c \in \mathbb{Z} a = 2, b = 4, c = 3$. I can compute $(2 + 4)3 = (2)(3) + (4)(3)$. Both simplify to $18$. From here it is trivialy true to see that $c(a + b) = ca + cb$. (R3) passes.
	\end{itemize}
	Therefore $(\mathbb{Z}, +, \cdot)$ is a ring.
	\item $(\mathbb{Z_n}, +, \cdot)$ Comparing this example with the previos one I can show $[a]_n + [b]_n = [a+b]_n$, and that $[a]_n \cdot [b]_n = [a \cdot b]_n$. Because I have this one to one map, it is valid to say that $(\mathbb{Z_n}, +, \cdot)$ is a ring as long as $(\mathbb{Z}, +, \cdot)$ is a ring, which has already been proved.
	\item $(\mathbb{U_n}, +, \cdot)$ Proof by counter example, in attempting to satisfy (R1) I found $2, 3 \in U_{10}$, but $2 + 3 = 5 \notin U_{10}$ since $gcd(5,10) \neq 1$.
	\item $(\mathbb{N}, +, \cdot)$ Proof by counter example, in attempting to satisfy (R1) I found that $(\mathbb{N}, +)$ does not contain any addative inverses. Specifically, $42 \in \mathbb{N}$, but $-42 \notin \mathbb{N}$.
	\item $\{a + b \sqrt{5} | a, b \in \mathbb{Z}\}$ This case is also proved in a similar fassion to the (1), but there are a couple special cases. $\sqrt{5} + \sqrt{5} = 2 \sqrt{5}$ which stays in the group. Also $\sqrt{5} \cdot \sqrt{5} = 5$ stays in the ring as well. Therefore, (5) is a ring.
	\item Intuitively this relation may seem like a ring, but the definition says nothing about it satisfying distributivity. So I have to go with No, it is not a ring.
\end{itemize}
\end{document}
