\documentclass{article}

% for \mathbb
\usepackage{amsfonts}
% aligning equations
\usepackage{amsmath}

\usepackage[margin=1in]{geometry}

\begin{document}
% Problem 1
\section{}
By definition $6x^{2} + 5x + 1 \equiv_{p} 0$ infers that
\[ p|6x^{2} + 5x + 1 \]
which validates the following for some $k \in \mathbb{Z}$
\[ pk = 6x^{2} + 5x + 1 \]
\[ 6x^{2} + 5x + 1 - pk = 0 \]
Now, assume I can factor the above equation into
\[ (3x - w)(2x + w) = 0 \]
For some $w \in \mathbb{Z}$ which satisfies both
\[ 3w - 2w = 5x \]
\[ w^{2} = 1 - pk \]
Then 

That 

Now, if I can find a $x \in \mathbb{Z}$ that satisfies $3w - 2w = 5x$ and $w^{2} = 1 - pk$ and  where $k$

% Problem 2
\section{}
\[
\begin{split}
% n1
n_{1} = 2, \quad
c_{1} = 1, \quad
m_{1} = 105, \quad
105d_{1} \equiv_{2} 1,\quad
d_{1} = 1 \\
% n2
n_{2} = 3, \quad
c_{2} = 2, \quad
m_{2} = 70, \quad
70d_{2} \equiv_{3} 1,\quad
d_{2} = 1 \\
% n3
n_{3} = 5, \quad
c_{3} = 0, \quad
m_{3} = 42, \quad
42d_{3} \equiv_{5} 1,\quad
d_{3} = 3 \\
% n4
n_{4} = 7, \quad
c_{4} = 3, \quad
m_{4} = 30, \quad
30d_{4} \equiv_{7} 1,\quad
d_{4} = 4 \\
\end{split}
\]
Hence
\[
\begin{split}
x \equiv_{210}
1 \cdot 105 \cdot 1 +
2 \cdot 70 \cdot 1 + 
0 \cdot 42 \cdot 3 +
3 \cdot 30 \cdot 4
\equiv_{210}
535
\equiv_{210}
115
\end{split}
\]
$x \equiv_{210} \textbf{115}$

% Problem 3
\section{}
$U_{15} = \{1, 2, 4, 7, 8, 11, 13, 14\}$
The order and generated group of each unit is
\[
\begin{split}
|1| = 1, \quad &\langle 1 \rangle = \{1\} \\
|2| = 4, \quad &\langle 2 \rangle = \{2,4,8,2\} \\
|4| = 2, \quad &\langle 4 \rangle = \{4, 1\} \\
|7| = 4, \quad &\langle 7 \rangle = \{7, 4, 13, 1\} \\
|8| = 4, \quad &\langle 8 \rangle = \{8, 4, 2, 1\} \\
|11| = 2, \quad &\langle 11 \rangle = \{11, 1\}\\
|13| = 4, \quad &\langle 13 \rangle = \{13, 4, 7, 1\}\\
|14| = 2, \quad &\langle 14 \rangle = \{14, 1\} \\
\end{split}
\]
No unit modulo $15$ is able to generate the entire $U_{15}$ group.
Therefore $U_{15}$ is not cyclic.

% Problem 4
\section{}
Since $n = 433$ is prime, $\phi(n) = N = 432 = 2^{4} \cdot 3^{3}$.
\[
\begin{split}
N_{1} = \frac{N}{2^{4}} = 27 &\quad N_{2} = \frac{N}{3^{4}} = 16
\end{split}
\]
Now I will compute $g_{1}, g_{2}, h_{1}, h_{2}$
\[
\begin{split}
g_{1} \equiv_{n} 7^{27} \equiv_{n} 265 \\
g_{2} \equiv_{n} 7^{16} \equiv_{n} 374 \\
h_{1} \equiv_{n} 166^{27} \equiv_{n} 250 \\
h_{2} \equiv_{n} 166^{16} \equiv_{n} 335 \\
\end{split}
\]
Iterating over $k$ until $g_{i}^k \equiv_{n} h_{i}$ is found
\newline
For $g_{1}$
\[
\begin{split}
265^{2} \equiv_{n} 153 \\
265^{3} \equiv_{n} 198 \\
... \\
265^{15} \equiv_{n} 250 \\
\end{split}
\]
For $g_{2}$
\[
\begin{split}
374^{2} \equiv_{n} 17 \\
374^{3} \equiv_{n} 296 \\
... \\
374^{20} \equiv_{n} 335 \\
\end{split}
\]
\[
\begin{split}
g_{1}^{15} \equiv_{n} 265^{15} \equiv_{n} 250 \equiv_{n} h_{1} &\quad \Rightarrow k_{1} = 15 \\
g_{2}^{20} \equiv_{n} 374^{20} \equiv_{n} 335 \equiv_{n} h_{2} &\quad \Rightarrow k_{2} = 20
\end{split}
\]
Using this result I will calculate $x$ using the Chinese Remainder theorem.
\[
\begin{cases}
x \equiv_{16} 15 \\
x \equiv_{27} 20 \\
\end{cases}
\]
\[
\begin{split}
x = 15 + 16y \\
15 + 16y \equiv_{27} 20 \Rightarrow & \\
&16y \equiv_{27} 5 \\
&y \equiv_{27} 5 \cdot 22 \\
&y \equiv_{27} 2 \\
x = 15 + 32 + 432k \\
\end{split}
\]
One such solution is $x = \textbf{47}$

% Problem 5
\section{}
Both $g^{a}$ and $g^{b}$ can be rewritten as 
\[
\begin{split}
(g^{m})^{a^{\prime}} \\
(g^{m})^{b^{\prime}} \\
\end{split}
\]
Where $m$ is the order of $g$ modulo $n$ and
$a^{\prime} = \frac{a}{m}$, 
$b^{\prime} = \frac{b}{m}$.
\newline
I know the values $a^{\prime}$ and $b^{\prime}$ exist because of this proof by contradiction.
\newline
Assume there is an $x$ such that $g^{x} \equiv_{n} 1$ but $m \not| x$ where $m = |g|$.
I then split $x$ into components as follows. $x = a_{0}m + a_{1}$ where $1 \le a_{1} < m$. Then
\[
\begin{split}
g^{a_{0}m + a_{1}} \equiv_{n} 1 & \\
&= g^{a_{0}m} \cdot g^{a_{1}} \equiv_{n} 1\\
&= (g^{m})^{a_{0}} \cdot g^{a_{1}} \equiv_{n} 1\\
&= 1^{a_{0}} \cdot g^{a_{1}} \equiv_{n} 1\\
&= g^{a_{1}} \equiv_{n} 1\\
\end{split}
\]
With $1 \le a_{1} < m$ this is a contradiction, because if such $a_{1}$ were to exist it would itself be the order.
\newline
Now that I know such an $a^{\prime}, b^{\prime} \in \mathbb{Z}$ and further that they share the divisor $m$,
then let $d = gcd(a, b)$. $m|d$, so let $d^{\prime} = \frac{d}{m}$ 


% Problem 6
\section{}
By expanding the defintion of $c_{1}$ and $c_{2}$ I get
\[
\begin{cases}
x \equiv_{p} mg_{1}^{s_{1}} \\
x \equiv_{q} mg_{2}^{s_{2}} \\
\end{cases}
\]
And then expanding $g_{1}$ and $g_{2}$ you get
\[
\begin{cases}
x \equiv_{p} m(g^{r_{1}(p-1)})^{s_{1}} \\
x \equiv_{q} m(g^{r_{2}(q-1)})^{s_{2}} \\
\end{cases}
\]
Then using proporties of exponentials
\[
\begin{cases}
x \equiv_{p} m(g^{s_{1}r_{1}})^{(p-1)} \\
x \equiv_{q} m(g^{s_{2}r_{2}})^{(q-1)} \\
\end{cases}
\]
Fermat's little theorem lets us cancel the $g^{...}$ term
\[
\begin{cases}
x \equiv_{p} m \\
x \equiv_{q} m \\
\end{cases}
\]
Because $p$ and $q$ are pairwise prime the solution to this Chinese Remainder will be $m$

\section{}
\subsection{}
\[
\begin{split}
1794677960^{(32411 - 1)/2} &\equiv_{32411} -1 \\
525734818^{(32411 - 1)/2} &\equiv_{32411} 1 \\
420526487^{(32411 - 1)/2} &\equiv_{32411} -1 \\
\end{split}
\]
Hence Alice's message is $1,0,1$
\subsection{}
$N = 3149 = 47 \cdot 67$
\[
\begin{split}
2322^{(47 - 1)/2} &\equiv_{47} -1\\
719^{(47 - 1)/2} &\equiv_{47} 1\\
202^{(47 - 1)/2} &\equiv_{47} 1\\
\end{split}
\]
Hence Alice's message is $1,0,0$
\subsection{}
\[
\begin{split}
(568980706 \cdot 705130839^{2}) \% 781044643 &= \textbf{517254876} \\
(568980706 \cdot 631364468^{2}) \% 781044643 &= \textbf{4308279} \\
(631364468^{2}) \% 781044643 &= \textbf{111914931} \\
\end{split}
\]
\section{}
Using the problem definition I will define $d, a^{\prime}, b^{\prime}$
\[
\begin{split}
d &= gcd(a, b) \\
a^{\prime} &= \frac{a}{d} \\ 
b^{\prime} &= \frac{b}{d} \\ 
\end{split}
\]
Consequently
\[
\begin{split}
gcd(a^{\prime}, b^{\prime}) &= 1 \\
(g^{d})^{a^{\prime}} &\equiv_{n} (g^{d})^{b^{\prime}} \\ 
\end{split}
\]
Using Lagranges Theorem, I know the only valid value for $g^{d} \equiv_{n} 1$

\section{}
because $n = 433$ is prime $\phi(n) = n-1 = 432 = 2^{4}\cdot 3^{3}$.
i will choose $n_{1} = 2^{4} = 16$, and $n_{2} = 3^{3} = 27$.
next i will compute $g_{1}, g{2}, h_{1}, h_{2}$
\[
\begin{split}
g_{1} = 7^{27} \equiv_{n}
\end{split}
\]

\section{}
\subsection{}
show that $n$ is a carmichael number
\newline
$n = 1729 = 7 \cdot 13 \cdot 19$ and hence it is composite. pick any $a$ coprime with $n$. by fermats little theorem
\[ \label{eq1}
\begin{split}
a^{6} \equiv_{7} 1   \\
a^{12} \equiv_{13} 1 \\
a^{18} \equiv_{19} 1 \\
\end{split}
\]
$n-1 = 1728 = 2^{6} \cdot 3^{3}$ so it is easy to see

\[
\begin{split}
1728/6 = 288 & \rightarrow a^{1278} \equiv_{7} (a^{6})^{288} \equiv_{7} 1 \\
1728/12 = 144 & \rightarrow a^{1278} \equiv_{13} (a^{12})^{144} \equiv_{13} 1 \\
1728/18 = 96 & \rightarrow a^{1278} \equiv_{19} (a^{18})^{96} \equiv_{19} 1 \\
\end{split}
\]
hence, $a^{n-1} \equiv_{n} 1$ and $n$ is carmichael
\subsection{}
$n-1 = 2^{6} \cdot 27$ so using $3$ as a base
\[
\begin{split}
3^{27} &\equiv_{1728} 664 \\
3^{2 \cdot 27} &\equiv_{1729} 1 \\
\end{split}
\]
returns \textbf{not prime}
\newline
since $3$ shows the compositeness of $1728$ it is a miller-rabin witness.
\section{}
using the given equiation
\[
\begin{split}
(763\cdot 773)^{2} \equiv_{n} (2^{6}\cdot 3^{3})^{2}
\end{split}
\]
hence
\[
\begin{split}
a = 763\cdot 773 &= 589799 \\
b = 2^{6}\cdot 3^{3} &= 1728 \\
\end{split}
\]
so $gcd(52907, a - b) = \textbf{277}$ is a non-trivial factor of $52907$
\subsection{}
i only have to check values $\phi(n)/p_{i}$ where $p_{i}$ is the $i^{th}$ prime.
\[
\begin{split}
\phi(113) = 112 = 2^{4} \cdot 7
\end{split}
\]
avoiding redundent values, i will only check $56$ and $16$ for $g = 2$
\[
\begin{split}
2^{16} \equiv_{113} 109
2^{56} \equiv_{113} 1
\end{split}
\]
hence, the order of $2$ mod $113$ is at least $56$ and therefore not a prime generator.
\subsection{}
using lagrange's theorem, the order of 3 must divide $\phi(n)$
\[
\begin{split}
3^{56} \equiv_{113} 112 \\
3^{16} \equiv_{113} 16 \\
\end{split}
\]
if the order of $3$ was anything but $\phi(n)$ one of congruences would have been congruent to $1$
\newline
hence the order of $3$ modulus $113$ is $\phi(113)$ or $112$

\end{document}
