\documentclass{article}

% for \mathbb
\usepackage{amsfonts}
% aligning equations
\usepackage{amsmath}

\usepackage[margin=1in]{geometry}

\begin{document}
Ben Lirio
\textit{I pledge my honor that I have abided by the Stevens Honor System}
% Problem 1
\section{}
First for $p = 2$
\[
\begin{split}
6 \cdot 1^{2} + 5 \cdot 1 + 1 &\equiv_{2} 0 \\
= 6 + 5 + 1 &\equiv_{2} 0 \\ 
= 12 &\equiv_{2} 0 \\ 
\end{split}
\]
And $p = 2$
\[
\begin{split}
6 \cdot 1^{2} + 5 \cdot 1 + 1 &\equiv_{3} 0 \\
= 6 + 5 + 1 &\equiv_{3} 0 \\ 
= 12 &\equiv_{3} 0 \\ 
\end{split}
\]
Now for $p \ne 3$ and $p \ne 2$
\[
\begin{split}
6x^{2} + 5x + 1 &\equiv_{p} 0 \\
x^{2} + \frac{5x}{6} + \frac{1}{6} &\equiv_{p} 0 \\
(x + \frac{5}{12})^{2} + \frac{1}{6} - \frac{25}{144} &\equiv_{p} 0 \\
(x + \frac{5}{12})^{2} &\equiv_{p} \frac{25}{144} - \frac{1}{6} \\
(x + \frac{5}{12})^{2} &\equiv_{p} \frac{1}{144} \\
x + \frac{5}{12} &\equiv_{p} \frac{1}{12} \\
x + \frac{5}{12} - \frac{1}{12}&\equiv_{p} 0 \\
x + \frac{4}{12} &\equiv_{p} 0 \\
x + \frac{1}{3} &\equiv_{p} 0 \\
\end{split}
\]
The following assumptions were used $\textrm{gcd}(3, p) = 1$ and $\textrm{gcd}(2, p) = 1$.
Which is true because $p$ is a prime and it is not $3$ or $2$.
\newline
For example $p = 23$, then $\frac{1}{3} = 8$. So $x \equiv_{23} -8 \equiv_{23} 15$.
I can then plug $x$ into the equation above
\[
\begin{split}
6 \cdot 15^{2} + 5 \cdot 15 + 1 &\equiv_{23} 0 \\
6 \cdot 225 + 5 \cdot 15 + 1 &\equiv_{23} 0 \\
11 \cdot 225 + 1 &\equiv_{23} 0 \\
11 \cdot 225 + 1 &\equiv_{23} 0 \\
2806 &\equiv_{23} 0 \\
\end{split}
\]
Is valid because $23|2806$

\newpage
% Problem 2
\section{}
\[
\begin{split}
% n1
n_{1} = 2, \quad
c_{1} = 1, \quad
m_{1} = 105, \quad
105d_{1} \equiv_{2} 1,\quad
d_{1} = 1 \\
% n2
n_{2} = 3, \quad
c_{2} = 2, \quad
m_{2} = 70, \quad
70d_{2} \equiv_{3} 1,\quad
d_{2} = 1 \\
% n3
n_{3} = 5, \quad
c_{3} = 0, \quad
m_{3} = 42, \quad
42d_{3} \equiv_{5} 1,\quad
d_{3} = 3 \\
% n4
n_{4} = 7, \quad
c_{4} = 3, \quad
m_{4} = 30, \quad
30d_{4} \equiv_{7} 1,\quad
d_{4} = 4 \\
\end{split}
\]
Hence
\[
\begin{split}
x \equiv_{210}
1 \cdot 105 \cdot 1 +
2 \cdot 70 \cdot 1 + 
0 \cdot 42 \cdot 3 +
3 \cdot 30 \cdot 4
\equiv_{210}
535
\equiv_{210}
115
\end{split}
\]
$x \equiv_{210} \textbf{115}$

\newpage
% Problem 3
\section{}
$U_{15} = \{1, 2, 4, 7, 8, 11, 13, 14\}$
The order and generated group of each unit is
\[
\begin{split}
|2| = 4, \quad &\langle 2 \rangle = \{2,4,8,2\} \\
|4| = 2, \quad &\langle 4 \rangle = \{4, 1\} \\
|7| = 4, \quad &\langle 7 \rangle = \{7, 4, 13, 1\} \\
|8| = 4, \quad &\langle 8 \rangle = \{8, 4, 2, 1\} \\
|11| = 2, \quad &\langle 11 \rangle = \{11, 1\}\\
|13| = 4, \quad &\langle 13 \rangle = \{13, 4, 7, 1\}\\
|14| = 2, \quad &\langle 14 \rangle = \{14, 1\} \\
\end{split}
\]
No unit modulo $15$ is able to generate the entire $U_{15}$ group.
Therefore $U_{15}$ is not cyclic.


\newpage
% Problem 4
\section{}
Both $g^{a}$ and $g^{b}$ can be rewritten as 
\[
\begin{split}
(g^{m})^{a^{\prime}} \\
(g^{m})^{b^{\prime}} \\
\end{split}
\]
Where $m$ is the order of $g$ modulo $n$ and
$a^{\prime} = \frac{a}{m}$, 
$b^{\prime} = \frac{b}{m}$.
\newline
I know the values $a^{\prime}$ and $b^{\prime}$ exist because of this proof by contradiction.
\newline
Assume there is an $x$ such that $g^{x} \equiv_{n} 1$ but $m \not| x$ where $m = |g|$.
I then split $x$ into components as follows. $x = a_{0}m + a_{1}$ where $1 \le a_{1} < m$. Then
\[
\begin{split}
g^{a_{0}m + a_{1}} \equiv_{n} 1 & \\
&= g^{a_{0}m} \cdot g^{a_{1}} \equiv_{n} 1\\
&= (g^{m})^{a_{0}} \cdot g^{a_{1}} \equiv_{n} 1\\
&= 1^{a_{0}} \cdot g^{a_{1}} \equiv_{n} 1\\
&= g^{a_{1}} \equiv_{n} 1\\
\end{split}
\]
With $1 \le a_{1} < m$ this is a contradiction, because if such $a_{1}$ were to exist it would itself be the order.
\newline
Now that I know such an $a^{\prime}, b^{\prime} \in \mathbb{Z}$ and further that they share the divisor $m$,
then let $d = gcd(a, b)$. $m|d$, so let $d^{\prime} = \frac{d}{m}$. The following equation now holds
\[
\begin{split}
g^{gcd(a,b)} \Rightarrow& \\
&= g^{d} \\
&= g^{md^{\prime}} \\
&= (g^{m})^{d^{\prime}} \\
&= (1)^{d^{\prime}} \\
&= 1 \\
\end{split}
\]
Done

\newpage
% Problem 5
\section{}
Since $n = 433$ is prime, $\phi(n) = N = 432 = 2^{4} \cdot 3^{3}$.
\[
\begin{split}
N_{1} = \frac{N}{2^{4}} = 27 &\quad N_{2} = \frac{N}{3^{4}} = 16
\end{split}
\]
Now I will compute $g_{1}, g_{2}, h_{1}, h_{2}$
\[
\begin{split}
g_{1} \equiv_{n} 7^{27} \equiv_{n} 265 \\
g_{2} \equiv_{n} 7^{16} \equiv_{n} 374 \\
h_{1} \equiv_{n} 166^{27} \equiv_{n} 250 \\
h_{2} \equiv_{n} 166^{16} \equiv_{n} 335 \\
\end{split}
\]
Iterating over $k$ until $g_{i}^k \equiv_{n} h_{i}$ is found
\newline
For $g_{1}$
\[
\begin{split}
265^{2} \equiv_{n} 153 \\
265^{3} \equiv_{n} 198 \\
... \\
265^{15} \equiv_{n} 250 \\
\end{split}
\]
For $g_{2}$
\[
\begin{split}
374^{2} \equiv_{n} 17 \\
374^{3} \equiv_{n} 296 \\
... \\
374^{20} \equiv_{n} 335 \\
\end{split}
\]
\[
\begin{split}
g_{1}^{15} \equiv_{n} 265^{15} \equiv_{n} 250 \equiv_{n} h_{1} &\quad \Rightarrow k_{1} = 15 \\
g_{2}^{20} \equiv_{n} 374^{20} \equiv_{n} 335 \equiv_{n} h_{2} &\quad \Rightarrow k_{2} = 20
\end{split}
\]
Using this result I will calculate $x$ using the Chinese Remainder theorem.
\[
\begin{cases}
x \equiv_{16} 15 \\
x \equiv_{27} 20 \\
\end{cases}
\]
\[
\begin{split}
x = 15 + 16y \\
15 + 16y \equiv_{27} 20 \Rightarrow & \\
&16y \equiv_{27} 5 \\
&y \equiv_{27} 5 \cdot 22 \\
&y \equiv_{27} 2 \\
x = 15 + 32 + 432k \\
\end{split}
\]
One such solution is $x = \textbf{47}$



\newpage
% Problem 6
\section{}
\subsection{}
By expanding the defintion of $c_{1}$ and $c_{2}$ I get
\[
\begin{cases}
x \equiv_{p} mg_{1}^{s_{1}} \\
x \equiv_{q} mg_{2}^{s_{2}} \\
\end{cases}
\]
And then expanding $g_{1}$ and $g_{2}$ you get
\[
\begin{cases}
x \equiv_{p} m(g^{r_{1}(p-1)})^{s_{1}} \\
x \equiv_{q} m(g^{r_{2}(q-1)})^{s_{2}} \\
\end{cases}
\]
Then using proporties of exponentials
\[
\begin{cases}
x \equiv_{p} m(g^{s_{1}r_{1}})^{(p-1)} \\
x \equiv_{q} m(g^{s_{2}r_{2}})^{(q-1)} \\
\end{cases}
\]
Fermat's little theorem lets us cancel the $g^{...}$ term
\[
\begin{cases}
x \equiv_{p} m \\
x \equiv_{q} m \\
\end{cases}
\]
Because $p$ and $q$ are pairwise prime the solution to this Chinese Remainder will be $m$
\subsection{}
The valnurability is in the distrabution of $g_{1}$ and $g_{2}$. Instead of being uniform from $1:n$, $g_{1}$ can only take at most $(q-1)$ values, and vise versa.
By looking at $g_{1}$ and $g_{2}$ like this
\[
\begin{split}
g_{1} \equiv_{n} (g^{(p-1)})^{r_{1}} \\
g_{2} \equiv_{n} (g^{(q-1)})^{r_{2}} \\
\end{split}
\]
we can immediatly see a vulnarabilty, the order $\big|g^{(p-1)}\big| = (q-1)$ and vise versa.
Furthermore, according to Lagrange's theorem, the order $\big|(g^{(p-1)})^{r_{1}}\big|\Big| (q-1)$.
By iterating over $i$ until $g_{1}^{i} \equiv_{n} 1$ we are able to get the order of $g_{1}$ which is a nontrivial
factor of $(q - 1)$. Doing the same with $g_{2}$ we can get a nontrivial factor of $(p - 1)$.

While this method works, the worst case running time is $2\cdot \sqrt{n}$. And if I am assuming I have that time, I could brute force by iterate from $1:\sqrt{n}$ attempting to divide $n$ by each. Yet, probablisticaly, $|g_{1}|$ may be far smaller than $\sqrt{n}$.

I wrote a program to break the code with $p = 101, q = 83$. With these small values, $|g_{1}| = 41$ and $|g_{2}| = 50$. Once I had these values, I was able to divide $n$ by $51, 101, ...$.

I am leaving it here, but I do believe there is a more efficient algorithm which is able to utilize both $g_{1}$ and $g_{2}$ and preform a chinese remainder theorem on congruences. 


\newpage
% Problem 7
\section{}
\subsection{}
\[
\begin{split}
1794677960^{(32411 - 1)/2} &\equiv_{32411} -1 \\
525734818^{(32411 - 1)/2} &\equiv_{32411} 1 \\
420526487^{(32411 - 1)/2} &\equiv_{32411} -1 \\
\end{split}
\]
Hence Alice's message is $1,0,1$
\subsection{}
$N = 3149 = 47 \cdot 67$
\[
\begin{split}
2322^{(47 - 1)/2} &\equiv_{47} -1\\
719^{(47 - 1)/2} &\equiv_{47} 1\\
202^{(47 - 1)/2} &\equiv_{47} 1\\
\end{split}
\]
Hence Alice's message is $1,0,0$
\subsection{}
\[
\begin{split}
(568980706 \cdot 705130839^{2}) \% 781044643 &= \textbf{517254876} \\
(568980706 \cdot 631364468^{2}) \% 781044643 &= \textbf{4308279} \\
(631364468^{2}) \% 781044643 &= \textbf{111914931} \\
\end{split}
\]

\newpage
\section{}
\subsection{}
$n = 1729 = 7 \cdot 13 \cdot 19$ therefore, $n$ is composite. 
\[
\begin{split}
a^{1728} \equiv_{7} a^{6} &\equiv_{7} 1   \\
a^{1728} \equiv_{7} a^{12} &\equiv_{13} 1 \\
a^{1728} \equiv_{7} a^{18} &\equiv_{19} 1 \\
\end{split}
\]
Which holds due to Fermat's little theorem, hence $n$ is a carmichael number.
\subsection{}
First I divide $n-1$ by $2$ until it is odd, producing $n-1 = 2^{6} \cdot 27$.
Next, using $3$ as a base, I will iterate over $3^{2^{i}\cdot 27}$
\[
\begin{split}
3^{27} &\equiv_{1728} 664 \\
3^{2 \cdot 27} &\equiv_{1729} 1 \\
\end{split}
\]
returns \textbf{not prime}
\newline
since $3$ shows the compositeness of $1728$ it is a miller-rabin witness.
\newpage
\section{}
using the given equiation
\[
\begin{split}
(763\cdot 773)^{2} \equiv_{n} (2^{6}\cdot 3^{3})^{2}
\end{split}
\]
hence
\[
\begin{split}
a = 763\cdot 773 &= 589799 \\
b = 2^{6}\cdot 3^{3} &= 1728 \\
\end{split}
\]
so $gcd(52907, a - b) = \textbf{277}$ is a non-trivial factor of $52907$
\newpage
\section{}
I only have to check values $\phi(n)/p_{i}$ where $p_{i}$ is the $i^{th}$ prime.
\[
\begin{split}
\phi(113) = 112 = 2^{4} \cdot 7
\end{split}
\]
avoiding redundent values, i will only check $56$ and $16$ for $g = 2$
\[
\begin{split}
2^{16} \equiv_{113} 109
2^{56} \equiv_{113} 1
\end{split}
\]
hence, the order of $2$ mod $113$ is less than or equal to $56$. Therefore $2$ modulo $113$ is not a prime generator.
\subsection{}
Using lagrange's theorem, the order of 3 must divide $\phi(n) = 112 = 2^{4} \cdot 7$.
\[
\begin{split}
3^{112/2} \equiv_{113} 3^{56} \equiv_{113} 112 \\
2^{112/7} \equiv_{113} 3^{16} \equiv_{113} 16 \\
\end{split}
\]
If $|3|$ modulo $113$ was anythig except $112$, I would have seen a congruence to $1$, as the following two number $56$ and $16$ contain all the factors in $112$. Therefore $|3| = \textbf{112}$ and furthermore is a prime generator.

\end{document}
