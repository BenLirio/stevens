\documentclass{article}

% for \mathbb
\usepackage{amsfonts}
% aligning equations
\usepackage{amsmath}
\usepackage{comment}

\usepackage[margin=1in]{geometry}

\begin{document}
Ben Lirio
\textit{I pledge my honor that I have abided by the Stevens Honor System}

% Problem 1
\section{}

After constructing a matrix with $x_1$, $x_2$, and $x_3$,
I will preform the basic row and column operations.
\[
\begin{bmatrix}
-2 & 4 & -1 \\
5 & 2 & 7 \\
4 & -3 & 2 \\
\end{bmatrix}
\rightarrow
\begin{bmatrix}
1 & 4 &  -2\\
-7 & 2 & 5 \\
-2 & -3 & 4 \\
\end{bmatrix}
\rightarrow
\begin{bmatrix}
1  & 0  & 0 \\
-7 & 30  & -9 \\
-2 & 5 & 0 \\
\end{bmatrix}
\rightarrow
\begin{bmatrix}
1  & 0  & 0 \\
0 & 30  & -9 \\
0 & 5 & 0 \\
\end{bmatrix}
\rightarrow
\begin{bmatrix}
1  & 0  & 0 \\
0 & 0  & -9 \\
0 & 5 & 0 \\
\end{bmatrix}
\rightarrow
\begin{bmatrix}
1  & 0  & 0 \\
0 & 5  & 0 \\
0 & 0 & 9 \\
\end{bmatrix}
\]
Thus $G \cong Z_{1} \times Z_{5} \times Z_{9} $

% Problem 2
\section{}
$F[x]$ is said to be a vector space when, for every $a, b \in F$ and $p(x), h(x) \in F[x]$
\begin{itemize}
\item $a(p(x) + h(x)) = ap(x) + ah(x)$
\item $(a + b)p(x) = ap(x) + bp(x)$
\item $a(bp(x)) = (ab)p(x)$
\item $1p(x) = p(x)$
\end{itemize}
\[
\begin{split}
p(x) &= c_0 + c_1x + c_2x^2 + ... + c_nx^n \\
h(x) &= d_0 + d_1x + d_2x^2 + ... + d_nx^n \\
\end{split}
\]
\[
\begin{split}
a(p(x) + h(x)) \\
&= a(c_0 + c_1x + c_2x^2 + ... + c_nx^n + d_0 + d_1x + d_2x^2 + ... + d_nx^n) \\
&= a((c_0 + d_0) + (c_1 + d_1)x + (c_2 + d_2)x^2 + ... + (c_n + d_n)x^n) \\
&= a(c_0 + d_0) + a(c_1 + d_1)x + a(c_2 + d_2)x^2 + ... + a(c_n + d_n)x^n \\
&= (ac_0 + ad_0) + (ac_1 + ad_1)x + (ac_2 + ad_2)x^2 + ... + (ac_n + ad_n)x^n \\
&= ac_0 + ac_1x + ac_2x^2 + ... + ac_nx^n + ad_0 + ad_1x + ad_2x^2 + ... + ad_nx^n \\
&= ap(x) + ah(x) \\
(a + b)p(x) \\
&= (a + b)(c_0 + c_1x + c_2x^2 + ... + c_nx^n) \\
&= (a + b)c_0 + (a + b)c_1x + (a + b)c_2x^2 + ... + (a + b)c_nx^n) \\
&= ac_0 + ac_1x + ac_2x^2 + ... + ac_nx^n + bc_0 + bc_1x + bc_2x^2 + ... + bc_nx^n \\
&= ap(x) + bp(x) \\
a(bp(x)) \\
&= a(b(c_0 + c_1x + c_2x^2 + ... + c_nx^n)) \\
&= a(bc_0 + bc_1x + bc_2x^2 + ... + bc_nx^n) \\
&= abc_0 + abc_1x + abc_2x^2 + ... + abc_nx^n \\
&= ab(c_0 + c_1x + c_2x^2 + ... + c_nx^n) \\
&= (ab)p(x) \\
1p(x) \\
&= 1(c_0 + c_1x + c_2x^2 + ... + c_nx^n) \\
&= 1c_0 + 1c_1x + 1c_2x^2 + ... + 1c_nx^n \\
&= c_0 + c_1x + c_2x^2 + ... + c_nx^n \\
&= p(x) \\
\end{split}
\]

% Problem 3
\section{}
\subsection{}
$K$ is an ideal in $F$ when,
for all $h^{\prime}(x) \in K$, and
for all $a(x) \in F[x]$,
\newline
then $a(x)h^{\prime}(x) \in K$,
\newline
\newline
From the definition that $K = I \cap J$, every $h^{\prime}(x) \in K$ satisfies
\begin{equation}\label{1}
\begin{split}
h^{\prime}(x) \in I \\
h^{\prime}(x) \in J \\
\end{split}
\end{equation}
Therefore, since $I$, and $J$ are both ideals in $F[x]$.
For all $a(x) \in F[x]$
\begin{equation}\label{2}
\begin{split}
f^{\prime}(x) \in I \implies a(x)f^{\prime}(x) \in I \\
g^{\prime}(x) \in J \implies a(x)g^{\prime}(x) \in J \\
\end{split}
\end{equation}
Then, using both $(1)$ and $(2)$, for all $h^{\prime}(x) \in K$, and $a(x) \in F[x]$
\[
\begin{split}
a(x)h^{\prime}(x) \in I \\
a(x)h^{\prime}(x) \in J \\
\end{split}
\]
Therefore, using $K = I \cap J$,
\[
a(x)h^{\prime}(x) \in K
\]
\subsection{}
Given
\[
\begin{split}
h(x) &\in K \\
\end{split}
\]
It is valid to claim,
\[
\begin{split}
h(x) &\in I \cap J \\
h(x) &\in I \\
h(x) &\in J \\
\end{split}
\]
Therefore, when $I = \langle f(x) \rangle$,
every element $f^{\prime}(x) \in I$ satisfies
\[
f^{\prime}(x) = a(x)f(x)
\]
for some $a(x) \in F[x]$ (the same is true for $g(x)$
\newline
\newline
Then, for some $a(x), b(x) \in F[x]$
\[
\begin{split}
h(x) = a(x)f(x) \\
h(x) = b(x)g(x) \\
\end{split}
\]
Which implies $f(x)$ and $h(x)$ both divide $h(x)$
\newline
Another way of saying this is, $h(x)$ is a common multiple of $f(x)$ and $g(x)$
\subsection{}
Proof by contradiction. Assume there exists some common multiple of $f(x)$ and $g(x)$ named $a(x)$ such that
\[
h(x) \not| a(x)
\]
Because $a(x)$ is a common multiple, for some $b(x), c(x) \in F[x]$
\[
\begin{split}
b(x)f(x) = a(x) \\
c(x)g(x) = a(x) \\
\end{split}
\]
Then, since every ideal of $F[x]$ is principle and
$\langle f(x) \rangle = I$, and $\langle g(x) \rangle = J$.
\[
\begin{split}
a(x) \in I \\
a(x) \in J \\
\end{split}
\]
Using the definitition $K = I \cap J$ and $\langle h(x) \rangle = K$
\[
a(x) \in K \implies  h(x) | a(x)
\]
This is again using the fact that every ideal in $F[x]$ is principle.
\newline
\newline
Comparing this the the assumption of $h(x) \not| a(x)$ we see there is a contradtion.

% Problem 4
\section{}
\subsection{}
If $x^3 + x^2 + 2x + 1 \in \mathbb{Z}_3[x]$ irreducible, then $E = \mathbb{Z}_3[x]/\langle f(x) \rangle$ is a field.
\newline
\newline
We can see $f(x) = x^3 + x^2 + 2x + 1 \in \mathbb{Z}_3[x]$ is irreducible by testing
\[
\begin{split}
f(0) = 0 + 0 + 0 + 1 \equiv_{3} 1 \ne 0 \\
f(1) = 1 + 1 + 2 + 1 \equiv_{3} 2 \ne 0 \\
f(2) = 8 + 4 + 4 + 1 \equiv_{3} 2 \ne 0 \\
\end{split}
\]
Becuase $f(x)$ does not contain any zeros in $\mathbb{Z}_3$ it is a field.
\subsection{}
\[
\begin{split}
\chi(E) &= 3 \\
|E| &= 3^3 = 27 \\
\end{split}
\]
\subsection{}
If $-x$ is primitive, then $(-x)^n\ \textrm{mod}\ x^3 + x^2 + 2x + 1 \ne 1$
for all natural numbers $n < 3^3-1$
\newline
\newline
In particular I only have to check the values $2$ and $13$ becuase
$\textrm{PFF}(26) = 2 \cdot 13$ and $\frac{26}{13} = 2$, $\frac{26}{2} = 13$.
\\
\\
Using WolframAlpha $\textrm{PolynomialMod}[(-x)^{13}, {x^3 + x^2 + 2x + 1}]$
\[
\begin{split}
(-x)^{2}\ \textrm{mod}\ x^3 + x^2 + 2x + 1 \equiv x^2 \\
(-x)^{13}\ \textrm{mod}\ x^3 + x^2 + 2x + 1 \equiv 1 \\
\end{split}
\]
Therefore, the order of $(-x)$ is $13$. \\
Hence, $(-x)$ is not primitive.
\subsection{}
The inverse of $(x+1) \in E$ is an element $a(x)$ such that
\[
(x+1) \cdot a(x) \equiv 1\ \textrm{mod}\ x^3 + x^2 + 2x + 1 \\
\]
And, since I know that $(x + 1)^{26} \equiv 1\ \textrm{mod}\ x^3+x^2+2x+1$ then,
\[
\begin{split}
(x+1) \cdot (x+1)^{25} \equiv 1\ \textrm{mod}\ x^3+x^2+2x+1
\end{split}
\]
Using WolframAlpha $\textrm{PolynomialMod}[(x+1)^{25}, {x^3+x^2+2x+1}]$
\[
(x+1)^{-1} \equiv (x^2 + 2)\ \textrm{mod}\ x^3+x^2+2x+1
\]
Technically $|(x+1)| = 13$ so I only have to calculate $(x+1)^{12}$,
but to be more general I kept $(x+1)^{25}$ (And becuase I have WolframAlpha).

% Problem 5
\section{}
Given the field $E = \mathbb{Z}_3[x]/\langle x^3 + x^2 + 2x + 1 \rangle$,
$g = x \in \mathbb{Z}_3[x]$,
and $h = x^2 + 2x + 2 \in \mathbb{Z}_3[x]$
I will compute $log_g(h) \in E$
using the \textbf{Pohlig-Hellman algorithm}.
\\
\\
Because the order of $|x| = 26$ in $E$, and the factors of $26$ are $2$ and $13$,
I will need two $N_i$'s.
\[
\begin{split}
N_1 = 26/2 = 13 \\
N_2 = 26/13 = 2 \\
\end{split}
\]
Next, I will use WolframAlpha to calculate $g_i$'s.
\[
\begin{split}
g_1 = x^{13} \equiv 2\ \textrm{mod}\ x^3 + x^2 + 2x + 1 \\
g_2 = x^2 \equiv x^2\ \textrm{mod}\ x^3 + x^2 + 2x + 1 \\
\end{split}
\]
Similarly, I will compute $h_i$'s.
\[
\begin{split}
h_1 = (x^2 + 2x + 2)^{13} \equiv 2\ \textrm{mod}\ x^3 + x^2 + 2x + 1 \\
h_2 = (x^2 + 2x + 2)^2 \equiv (x+1)\ \textrm{mod}\ x^3 + x^2 + 2x + 1 \\
\end{split}
\]
Then through simple iteration with WolframAlpha, I get.
\[
\begin{split}
log_{2}(2) = 1 = x_1 \\
log_{x^2}(x+1) = 4 = x_2 \\
\end{split}
\]
Making sure to keep my primes in the same order $2$ then $13$,
I create the following congruence
\[
\begin{cases}
	x \equiv_2 1 \\
	x \equiv_{13} 4 \\
\end{cases}
\]
Then a simple Chinese Remainder Algorithm
\[
\begin{split}
x &\equiv_2 1 \\
x &= 2y + 1 \\
2y + 1 &\equiv_{13} 4 \\
2y &\equiv_{13} 3 \\
y &\equiv_{13} 8 \\
x &= 2 \cdot 8 + 1 \\
x &= 17 \\
\end{split}
\]
Therefore, $x^{17} \equiv x^2 + 2x + 2\ \textrm{mod}\ x^3 + x^2 + 2x + 1$
\\
\\
In $\mathbb{Z}_3[x]/\langle x^3 + x^2 + 2x + 1 \rangle $, $log_{x}(x^2 + 2x + 2) = 17$
% Problem 6
\section{}
\subsection{}
Given polynomials 
$f(x) = 2x^3 + 6x^2 + 5x + 1$ and 
$g(x) = 3x^4 + x^3 + 3x^2 + x + 3$ both in 
$\mathbb{Z}_7[x]$, 
I will use the Extended Euclidean Algorithm to find $\textrm{gcd}\big(g(x), f(x)\big)$
\[
\begin{split}
g(x) = f(x) \cdot (5x + 3) + (2x^2 + 2x) &\implies \textrm{gcd}\big(g(x), f(x)\big) = \textrm{gcd}\big(f(x), (2x^2 + 2x)\big) \\ \\
f(x) = (2x^2 + 2x) \cdot (x + 2) + (x + 1) &\implies \textrm{gcd}\big(f(x), (2x^2 + 2x)\big) = \textrm{gcd}\big((2x^2+2x),(x+1)\big) \\ \\
(2x^2 + 2x) = (x+1) \cdot (2x) + 0 &\implies \textrm{gcd}\big((2x^2+2x),(x+1)\big) = \textrm{gcd}\big((x+1), 0\big) = (x+1)\\
\end{split}
\]
\\
Therefore $\textrm{gcd}\big(f(x), g(x)\big) = (x+1)$
\subsection{}
Here, I will compute $\alpha(x), \beta(x) \in \mathbb{Z}_7[x]$, such that
$\textrm{gcd}\big(f(x), g(x)\big) = \alpha(x)f(x) + \beta(x)g(x)$.
\[
\begin{split}
(x+1) &= f(x) - (2x^2+2x)\cdot(x+2) \\
(2x^2+2x) &= g(x) - f(x)\cdot(5x+3) \\
(x+1) &= f(x) - \big(g(x) - f(x)\cdot(5x+3)\big)\cdot(x+2) \\
(x+1) &= f(x)\cdot(1 + (5x+3)\cdot(x+2))-g(x)\cdot(x+2) \\
(x+1) &= f(x)\cdot(5x^2+6x) + g(x)\cdot -(x+2) \\
(x+1) &= f(x)\cdot(5x^2+6x) + g(x)\cdot (6x+5) \\
\end{split}
\]
Hence, $\alpha(x) = (5x^2+6x)$ and $\beta(x) = (6x+5)$
% Problem 7
\section{}
\subsection{}
Given $f(x) = x^3 + 1 \in \mathbb{Z}_2[x]$, the multiplication table of the
quotient ring $E = \mathbb{Z}_2[x]/ \langle f(x)\rangle$ is
\begin{center}
\begin{tabular}{ c | c | c | c | c | c | c | c | c }
& $0$ & $1$ & $x$ & $x + 1$ & $x^2$ & $x^2 + 1$ & $x^2 + x$ & $x^2 + x + 1$ \\
\hline
$0$ & $0$ & $0$ & $0$ & $0$ & $0$ & $0$ & $0$ & $0$ \\
\hline
$1$ & $0$ & $1$ & $x$ & $x + 1$ & $x^2$ & $x^2 + 1$ & $x^2 + x$ & $x^2 + x + 1$ \\
\hline
$x$ & $0$ & $x$ & $x^2$ & $x^2 + x$ & $1$ & $x + 1$ & $x^2 + 1$ & $x^2 + x + 1$ \\
\hline
$x+1$ & $0$ & $x+1$ & $x^2 + x$ & $x^2 + 1$ & $x^2 + 1$ & $x^2 + x$ & $x^2 + 1$ & $0$ \\
\hline
$x^2$ & $0$ & $x^2$ & $1$ & $x^2 + 1$ & $x$ & $x^2 + x$ & $x + 1$ & $x^2 + x + 1$ \\
\hline
$x^2 + 1$ & $0$ & $x^2 + 1$ & $x+1$ & $x^2 + x$ & $x^2+x$ & $x + 1$ & $x^2 + 1$ & $0$ \\
\hline
$x^2 + x$ & $0$ & $x^2 + x$ & $x^2+1$ & $x + 1$ & $x+1$ & $x^2 + 1$ & $x^2 + x$ & $0$ \\
\hline
$x^2 + x + 1$ & $0$ & $x^2 + x + 1$ & $x^2+x+1$ & $0$ & $x^2+x+1$ & $0$ & $0$ & $x^2+x+1$ \\
\end{tabular}
\end{center}
\subsection{}
The zero divisors of $E$ are
\[
\begin{split}
(x+1) \cdot (x^2+x+1) \equiv 0 &\implies (x+1)\ \textrm{is a zero divisor} \\
(x^2+1) \cdot (x^2+x+1) \equiv 0 &\implies (x^2+1)\ \textrm{is a zero divisor} \\
(x^2+x) \cdot (x^2+x+1) \equiv 0 &\implies (x^2+x)\ \textrm{is a zero divisor} \\
(x^2+x+1) \cdot (x+1) \equiv 0 &\implies (x^2+x+1)\ \textrm{is a zero divisor} \\
\end{split}
\]
\subsection{}
The set of units $U$ in $E$ are
\[
\begin{split}
(1)\cdot(1) \equiv 1 &\implies (1) \in U \\
(x)\cdot(x^2) \equiv 1 &\implies (x) \in U \\
(x^2)\cdot(x) \equiv 1 &\implies (x^2) \in U \\
\end{split}
\]
\subsection{}
The unit $U$ in $E$ are closed under multiplication
\begin{center}
\begin{tabular}{ c | c | c | c }
& $1$ & $x$ & $x^2$ \\
\hline
$1$ & $1$ & $x$ & $x^2$ \\
\hline
$x$ & $x$ & $x^2$ & $1$ \\
\hline
$x^2$ & $x^2$ & $1$ & $x$ \\
\end{tabular}
\end{center}
\subsection{}
The set $U$ is a group under $\cdot$ because it satisfies
\\
\\
\textbf{Associativity}. $a(bc) = (ab)c$ for all $a,b,c \in E$.
\\
\\
\textbf{Identity}. The identity $e = 1$ in $U$.
\\
\\
\textbf{Inverses}. The set $U$ is defined as the elements in $E$ with an inverse.
\subsection{}
The primitive elements in the group $U$ are elements that generate $U$
\[
\begin{split}
(x) &= (x) \\
(x)^2 &= (x^2) \\
(x)^3 &= (1) \\
\\
(x^2) &= (x^2) \\
(x^2)^2 &= (x) \\
(x^2)^3 &= (1) \\
\end{split}
\]
$U$ contains $2$ primitive elements.

% Problem 8
\section{}
To find the dishonest participant, I will compute the secret with two keys at 
a time until I have enough information
\\
\\
First, with $(12, 2)$ and $(3, 14)$
\[
\begin{split}
l_1(x) &= \frac{x-14}{2-14} = 7(x-14) = 7x+4 \\
l_2(x) &= \frac{x-2}{14-2} = 10(x-2) = 10x+14 \\
\end{split}
\]
\[
12\cdot (7x+4) + 3 \cdot (10x+14) = 12x+5 \\
\]
Using share $\#1, \#2$, the secret is $5$.
\\
\\
Then, with $(12, 2)$ and $(9, 11)$
\[
\begin{split}
l_1(x) &= \frac{x-11}{2-11} = 15(x-11) = 15x+5 \\
l_2(x) &= \frac{x-2}{11-2} = 2(x-2) = 2x+13 \\
\end{split}
\]
\[
12\cdot (15x+5) + 9 \cdot (2x+13) = 11x+7 \\
\]
Using share $\#1, \#3$, the secret is $7$.
\\
\\
Now I know that either $\#1, \#2,$ or $\#3$ is a bad share.
\\
Therefore, I know $\#4$ is correct
\\
\\
Using $(3, 14)$ and $(7, 12)$
\[
\begin{split}
l_1(x) &= \frac{x-12}{14-12} = 9(x-12) = 9x+11\\
l_2(x) &= \frac{x-14}{12-14} = 8(x-14) = 8x+7\\
\end{split}
\]
\[
3\cdot (9x+11) + 7 \cdot (8x+7) = 15x+14 \\
\]
Using $\#4$ and $\#2$ I get $14$ as my secret
\\
\\
If $\#4$ is correct and none of my answers match, that must mean $\#1$ is dishonest
which makes $14$ my secret.
\\
\\
To verify I will also check $\#2$ and $\#3$
\\
With $(3, 14)$ and $(9, 11)$ I get
\[
\begin{split}
l_1(x) &= \frac{x-11}{14-11} = 6(x-11) = 6x+2\\
l_2(x) &= \frac{x-14}{11-14} = 11(x-14) = 11x+16\\
\end{split}
\]
\[
3\cdot (6x+2) + 9 \cdot (11x+16) = 15x +  14\\
\]
The secret of $14$ is verified.
% Problem 9
\section{}
\subsection{}
The elliptic curve $y^2 = x^3 + 2x + 6$ is not singular because
\[
4\cdot 2^3 + 27 \cdot 6^2 \ne 0 
\]
\subsection{}
To find the order of $(1, 3) \in \mathcal{E}$ can be calculated by
taking the multiples of $(1, 3)$
\[
\begin{split}
1\cdot(1,3) = (1,3) \\
2\cdot(1,3) = (7,5) \\
3\cdot(1,3) = (8,12) \\
4\cdot(1,3) = (3,0) \\
5\cdot(1,3) = (8,1) \\
6\cdot(1,3) = (7,8) \\
7\cdot(1,3) = (1,10) \\
8\cdot(1,3) = \mathcal{O} \\
\end{split}
\]
Hence, $|(1, 3)| = 8$
\subsection{}
$\mathcal{E}$ is cyclic over $\mathbb{Z}_{13}$ if and only if 
$\textrm{gcd}\big(|\mathcal{E}|, 13-1\big) = 1$
\[
\textrm{gcd}\big(16, 12\big) \ne 1 
\]
Hence, $\mathcal{E}$ is not cyclic.
\\
\\
To demonstrate this, here is the order of each element
\[
\begin{split}
|(1,10)| = 8 \\
|(3,0)| = 2 \\
|(4,0)| = 2 \\
|(6,0)| = 2 \\
|(7,5)| = 4 \\
|(7,8)| = 4 \\
|(8,1)| = 6 \\
|(8,12)| = 8 \\
|(9,5)| = 8 \\
|(9,8)| = 8 \\
|(10,5)| = 8 \\
|(10,8)| = 8 \\
|(12,4)| = 4 \\
|(12,9)| = 4 \\
\end{split}
\]
For all $(x,y) \in \mathcal{E}$, $\langle (x,y)\rangle \ne \mathcal{E}$.
\subsection{}
Given $g = (1,3)$, $a \cdot g = (8, 12)$, and $b \cdot g = (7, 8)$, I will compute
$K = a\cdot B = b \cdot A$.
\\
\\
Using the answer from part $(b)$, I can compute the 
$\textrm{DLOG}(g, a\cdot g)$, and 
$\textrm{DLOG}(g, b\cdot g)$.
\[
\begin{split}
\textrm{DLOG}\big((1,3), (8, 12)\big) = 3 \\
\textrm{DLOG}\big((1,3), (7, 8)\big) = 6 \\
\end{split}
\]
By finding the discrete log of both Alice and Bobs public key,
I can find thier shared key as follows
\[
\begin{split}
3 \cdot (7, 8) = 6 \cdot (8, 12) = (7, 5)
\end{split}
\]
Alice and Bobs shared key is $(7, 5)$
% Problem 10
\section{}
Given elliptic curve $\mathcal{E}$ defined by the equation $y^2 = x^3 + 3x + 4$ over $\mathbb{Z}_{11}$
and the following information
\[
\begin{split}
g = (5, 1) \\
A = (9, 10) \\
c1 = (7, 7) \\
c2 = (9, 1) \\
\end{split}
\]
I can compute $m$ with the equation $m = c2 - a\cdot c1$.
\\
\\
First, using the multiples of $g$ shown below, I can find $a$ in the equation $A = a\cdot g$
\[
(9, 10) = 5 \cdot (5, 1)
\]
Because I have the  multiples of $g$ computed, I well find $j$ in the equation 
$c1 = j\cdot g$ so that I can compute $a\cdot c1 = (aj)\cdot g$.
\[
(7, 7) = 3 \cdot (5, 1)
\]
Finally, $m = c2 - a\cdot c1$ can be computed as follows
\[
\begin{split}
m &=c2 - a\cdot c1 \\
&=c2 - (aj)\cdot g \\
&=(9,1) - (3 \cdot 5)\cdot (5, 1) \\
&=(9,1) - (15)\cdot (5, 1) \\
&=(9,1) - (5, 1) \\
&=9\cdot(5,1) + -1\cdot(5, 1) \\
&=8\cdot(5,1) \\
&=(0, 9)
\end{split}
\]
$m = (0, 9)$
\\
\\
multiples of $g$
\[
\begin{split}
g = (5, 1) \\
2\cdot g = (4, 5) \\
3\cdot g = (7, 7) \\
4\cdot g = (8, 1) \\
5\cdot g = (9, 10) \\
6\cdot g = (0, 2) \\
7\cdot g = (10, 0) \\
8\cdot g = (0, 9) \\
9\cdot g = (9, 1) \\
10\cdot g = (8, 10) \\
11\cdot g = (7, 4) \\
12\cdot g = (4, 6) \\
13\cdot g = (5, 10) \\
14\cdot g = \mathcal{O}\\
\end{split}
\]
Here are the first couple computations of $g$, but I wrote a program to do the rest
\\
$2\cdot g$
\[
\begin{split}
(x_1, y_1) = (x_2, y_2) = (5, 1) \\
\end{split}
\]
$x_1 = x_2$ and $y_1 = y_2$ means I will use Case II
\[
\begin{split}
\lambda &= \frac{3\cdot (5)^2 + 3}{2\cdot 1} \\
&= \frac{3\cdot 3 + 3}{2} \\
&= \frac{1}{2} \\
&= 6\cdot 1\\
&= 6\\
\end{split}
\]
Next, I will find $x_3 = \lambda^{2} - x_1 - x_2$
\[
\begin{split}
x_3 &= 6^2 - 5 - 5 \\
&= 3 - 10 \\
&= 3 + 1 \\
&= 4
\end{split}
\]
And for $y_3 = \lambda \cdot(x_1 - x_3) - y_1$
\[
\begin{split}
y_3 &= 6(5 - 4) - 1 \\
&= 6 - 1 \\
&= 5
\end{split}
\]
$2\cdot g = (4, 5)$
\\
Then I will find $3\cdot g$ by computing $g + 2\cdot g$
\[
\begin{split}
(x_1, y_1) = (4, 5) \\
(x_2, y_2) = (5, 1) \\
\end{split}
\]
$x_1 \ne x_2$ means I will use Case I
\[
\begin{split}
\lambda &= \frac{1 - 5}{5 - 4} \\
&= \frac{7}{1} \\
&= 7\\
\end{split}
\]
Next, I will find $x_3 = \lambda^{2} - x_1 - x_2$
\[
\begin{split}
x_3 &= 7^2 - 4 - 5 \\
&= 5 - 9 \\
&= 5 + 2 \\
&= 7
\end{split}
\]
And for $y_3 = \lambda \cdot(x_1 - x_3) - y_1$
\[
\begin{split}
y_3 &= 7(4 - 7) - 5 \\
&= 12 - 5 \\
&= 7
\end{split}
\]
$3\cdot g = (7, 7)$
\\
Code to generate multiples of $g$ written in \textbf{golang}
\begin{verbatim}
package main

import (
	"math"
	"fmt"
)

var m int = 11
var a int = 3
var b int = 4

func mod(v int) int {
	out := int(math.Mod(float64(v), float64(m)))
	if out < 0 {
		out += m
	}
	return out
}

func inv(v int) int {
	for i := 0; i < m; i++ {
		if mod(i*v) == 1 {
			return i
		}
	}
	return -1
}
func pow(b int, p int) int {
	out := 1
	for i := 0; i < p; i++ {
		out *= b
	}
	return out
}

func add(x1 int, y1 int, x2 int, y2 int) (bool, int, int) {
	if (x1 == x2) && (y2 != y1) {
		return true, 0, 0
	}
	if (x1 == x2) && (y1 == 0) && (y2 == 0) {
		return true, 0, 0
	}
	var lambda int
	if (x1 == x2) {
		top := mod(3*pow(x1,2) + a)
		bottom := mod(2*y1)
		lambda = mod(top*inv(bottom))
	} else {
		top := mod(y2 - y1)
		bottom := mod(x2 - x1)
		lambda = mod(top*inv(bottom))
	}
	x3 := mod(pow(lambda, 2)-x1-x2)
	y3 := mod(lambda*(x1-x3) - y1)
	return false, x3, y3
}

func main() {
	gX := 7
	gY := 7
	curX := 7
	curY := 7
	var check bool
	for {
		check, curX, curY = add(curX, curY, gX, gY)
		if check {
			break
		}
		fmt.Println(curX, curY)
	}
}
\end{verbatim}
\end{document}
