\documentclass{article}
\usepackage{amsmath}
\usepackage{bm}
\begin{document}
\begin{titlepage}
	\vspace*{\stretch{1.0}}
		\begin{center}
		\Large\textbf{Ben Lirio}\\
		\vspace*{\stretch{.1}}
		\large{3/12/21} \\
		\vspace*{\stretch{.1}}
		\large\textit{I pledge my honor that I have abided by the Steven Honor System.}
		\end{center}
	\vspace*{\stretch{2.0}}
\end{titlepage}
\section{Problem}
Find the line $y = C + Dx$ that best fits the data 
$(x, y) = \{(-2, 4), (-1, 2), (0, -1), (1, 0), (2, 0)\}$

In matrix form, this can be viewed as
\[
\begin{bmatrix}
1 & -2 \\
1 & -1 \\
1 & 0 \\
1 & 1 \\
1 & 2 \\
\end{bmatrix}
\cdot
\begin{bmatrix}
C \\
D \\
\end{bmatrix}
=
\begin{bmatrix}
4 \\
2 \\
-1 \\
0 \\
0 
\end{bmatrix}
\]
Using the formula $A^{T}A\hat{x} = A^{t}b$
\[
\begin{bmatrix}
1 & 1 & 1 & 1 & 1 \\
-2 & -1 & 0 & 1 & 2 \\
\end{bmatrix}
\cdot
\begin{bmatrix}
1 & -2 \\
1 & -1 \\
1 & 0 \\
1 & 1 \\
1 & 2 \\
\end{bmatrix}
\cdot
\begin{bmatrix}
C \\
D \\
\end{bmatrix}
=
\begin{bmatrix}
1 & 1 & 1 & 1 & 1 \\
-2 & -1 & 0 & 1 & 2 \\
\end{bmatrix}
\cdot
\begin{bmatrix}
4 \\
2 \\
-1 \\
0 \\
0 
\end{bmatrix}
\]

\[
\begin{bmatrix}
5 & 0 \\
0 & 10 \\
\end{bmatrix}
\cdot
\begin{bmatrix}
C \\
D \\
\end{bmatrix}
=
\begin{bmatrix}
5 \\
-10 \\
\end{bmatrix}
\]

\[
\begin{bmatrix}
\frac{1}{5} & 0 \\
0 & \frac{1}{10} \\
\end{bmatrix}
\begin{bmatrix}
5 & 0 \\
0 & 10 \\
\end{bmatrix}
\begin{bmatrix}
C \\
D \\
\end{bmatrix}
=
\begin{bmatrix}
\frac{1}{5} & 0 \\
0 & \frac{1}{10} \\
\end{bmatrix}
\begin{bmatrix}
5 \\
-10 \\
\end{bmatrix}
\]

\[
\begin{bmatrix}
C \\
D \\
\end{bmatrix}
=
\begin{bmatrix}
1 \\
-1 \\
\end{bmatrix}
\]

Which can be interpreted as $y = 1 + (-1)x$


\section{Problem}
Use the Gram-Schmidt method to find orthonormal vectors
\newline
$A = a$
\newline
First I will find the component of $b$ that is orthogonal to $A$ by solving $e = b - p$ where $p$ is the projection vector. In order to do so, I will have to compute $P = A(A^{T}\cdot A)^{-1}A^{T}$, the projection matrix of $A$ and use it to find $p = Pb$.

\[
P
=
\begin{bmatrix}
1 \\
-1 \\
0 \\
0 \\
\end{bmatrix}
\Bigg(
\begin{bmatrix}
1 & -1 & 0 & 0 \\
\end{bmatrix}
\begin{bmatrix}
1 \\
-1 \\
0 \\
0 \\
\end{bmatrix}
\Bigg)^{-1}
\begin{bmatrix}
1 & -1 & 0 & 0 \\
\end{bmatrix}
=
\begin{bmatrix}
.5 & -.5 & 0 & 0 \\
-.5 & .5 & 0 & 0 \\
0 & 0 & 0 & 0  \\
0 & 0 & 0 & 0  \\
\end{bmatrix}
\]

\[
p
=
\begin{bmatrix}
.5 & -.5 & 0 & 0 \\
-.5 & .5 & 0 & 0 \\
0 & 0 & 0 & 0  \\
0 & 0 & 0 & 0  \\
\end{bmatrix}
\cdot
\begin{bmatrix}
0 \\
1 \\
-1 \\
0 \\
\end{bmatrix}
=
\begin{bmatrix}
-.5 \\
.5 \\
0 \\
0 \\
\end{bmatrix}
\]

\[
e
=
\begin{bmatrix}
0 \\
1 \\
-1 \\
0
\end{bmatrix}
-
\begin{bmatrix}
-.5 \\
.5 \\
0 \\
0 \\
\end{bmatrix}
=
\begin{bmatrix}
.5 \\
.5 \\
-1 \\
0 \\
\end{bmatrix}
\]
Now I will set $B = e$
\newline
In order to get $C$ I will construct a new matrix $Q$ which contains two orthogonal colums where $q_{1} = A$ and $q_{2} = B$
Then I can set $C$ to be the part of $c$ that is orthogonal to $Q$.
\[
P
=
\begin{bmatrix}
1 & .5 \\
-1 & .5 \\
0 & -1 \\
0 & 0\\
\end{bmatrix}
\Bigg(
\begin{bmatrix}
1 & -1 & 0 & 0 \\
.5 & .5 & -1 & 0 \\
\end{bmatrix}
\cdot
\begin{bmatrix}
1 & .5 \\
-1 & .5 \\
0 & -1 \\
0 & 0\\
\end{bmatrix}
\Bigg)^{-1}
\begin{bmatrix}
1 & -1 & 0 & 0 \\
.5 & .5 & -1 & 0 \\
\end{bmatrix}
=
\begin{bmatrix}
\frac{2}{3} & -\frac{1}{3} & \frac{1}{3} & 0 \\
-\frac{1}{3} & \frac{2}{3} & -\frac{1}{3} & 0 \\
-\frac{1}{3} & -\frac{1}{3} & \frac{2}{3} & 0 \\
0 & 0 & 0 & 0 \\
\end{bmatrix}
\]

\[
p
=
P\cdot c
=
\begin{bmatrix}
\frac{2}{3} & -\frac{1}{3} & \frac{1}{3} & 0 \\
-\frac{1}{3} & \frac{2}{3} & -\frac{1}{3} & 0 \\
-\frac{1}{3} & -\frac{1}{3} & \frac{2}{3} & 0 \\
0 & 0 & 0 & 0 \\
\end{bmatrix}
\begin{bmatrix}
0 \\
0 \\
1 \\
-1 \\
\end{bmatrix}
=
\begin{bmatrix}
-1/3 \\
-1/3 \\
2/3 \\
0 \\
\end{bmatrix}
\]

\[
e
=
\begin{bmatrix}
0 \\
0 \\
1 \\
-1 \\
\end{bmatrix}
-
\begin{bmatrix}
-1/3 \\
-1/3 \\
2/3 \\
0 \\
\end{bmatrix}
=
\begin{bmatrix}
1/3 \\
1/3 \\
1/3 \\
-1 \\
\end{bmatrix}
\]

Now Let $C = e$
\newline
Next I have to normalize the vectors, so set $A \leftarrow \frac{A}{||A||}$ and so on for all vectors.
$||A|| = \sqrt{1 + 1} = \sqrt{2}$
\newline
$||B|| = \sqrt{1/4 + 1/4 + 1} = \sqrt{3/2} = \frac{\sqrt{6}}{2}$
\newline
$||C|| = \sqrt{1/9 + 1/9 + 1/9 + 1} = \sqrt{2/3} = \frac{2\cdot \sqrt{3}}{3}$
\newline
\[
\begin{bmatrix}
\frac{1}{\sqrt{2}} & \frac{2}{2\cdot \sqrt{6}} & \frac{3}{6\sqrt{3}} \\
\\
-\frac{1}{\sqrt{2}} & \frac{2}{2\cdot \sqrt{6}} & \frac{3}{6\sqrt{3}} \\
\\
0 & \frac{2}{2\cdot \sqrt{6}} & \frac{3}{6\sqrt{3}} \\
\\
0 & 0 & -\frac{3}{2\sqrt{3}} \\
\end{bmatrix}
\]
I know it looks a bit messy, but it works.

\section{Problem}
Supose $Q_{1}, Q_{2}$ are square $n x n$ matrices that are orthonormal. Show that their product $Q_{1}Q_{2}$ is orthonormal square matrix.

Let $Q_{3} = Q_{1}Q_{2}$. Now iff $Q_{3}$ is a non trivial orthonoraml matrix then $Q_{3}^{T}Q_{3} = I$.
\newline
\[
Q_{3}^{T}Q_{3} \\
= (Q_{1}Q_{2})^{T}(Q_{1}Q_{2}) \\
= Q_{2}^{T}(Q_{1}^{T}Q_{1})Q_{2} \\
= Q_{2}^{T}Q_{2} \\
= I
\]
Therefore, $Q_{3}$ must be an orthonormal matrix.

\section{Problem}
Let $A, B, C, D$ be $2 x 2$ matrices. Does the following equality always hold? (If yes prove why, if not find a counterexample)
\[
	det(\begin{bmatrix}A & B\\C & D\end{bmatrix}) = det(A) \cdot det(D) - det(C) \cdot det(B)
\]
Counter Example
\[
\begin{bmatrix}
1 & 0 & -1 & -1 \\
-1 & 1 & -1 & 0 \\
0 & 1 & 1 & 0 \\
0 & -1 & 1 & 0 \\
\end{bmatrix}
\]
This example has a determinant of 2 but $det(D) = det(C) = 0$, therefore the equation above will be zero no matter what determinant $A$ and $B$ have.


\section{Problem}
Reduce $A = $
\[
\begin{bmatrix}
1 & 1 & 1 \\
1 & 2 & 2 \\
1 & 2 & 3   
\end{bmatrix}
\]
\[
\begin{bmatrix}
1 & 1 & 1 \\
0 & 1 & 1 \\
0 & 1 & 2   
\end{bmatrix}
\]
\[
U
=
\begin{bmatrix}
1 & 1 & 1 \\
0 & 1 & 1 \\
0 & 0 & 1   
\end{bmatrix}
\]
Determinant of A is $1 \cdot 1 \cdot 1 = 1$
\end{document}

